%%% -*-LaTeX-*-
%%% riskgame.tex.orig
%%% Prettyprinted by texpretty lex version 0.02 [21-May-2001]
%%% on Fri Jun 17 09:16:08 2022
%%% for Steve Dunbar (sdunbar@family-desktop)

\documentclass[12pt]{article}

\input{../../../../etc/macros} %\input{../../../../etc/mzlatex_macros}
\input{../../../../etc/pdf_macros}

\bibliographystyle{plain}

\begin{document}

\myheader \mytitle

\hr

\sectiontitle{The Game of \emph{Risk} as a Markov Chain}

\hr

\usefirefox

\hr

% \visual{Study Tip}{../../../../CommonInformation/Lessons/studytip.png}
% \section*{Study Tip}

% \hr

\visual{Rating}{../../../../CommonInformation/Lessons/rating.png}
\section*{Rating} %one of
% Everyone: contains no mathematics.
% Student: contains scenes of mild algebra or calculus that may require guidance.
Mathematically Mature:  may contain mathematics beyond calculus with
proofs.  % Mathematicians Only: prolonged scenes of intense rigor.

\hr

\visual{Section Starter Question}{../../../../CommonInformation/Lessons/question_mark.png}
\section*{Section Starter Question}

When rolling three fair six-sided dice with outcomes \( Y_1 \), \( Y_2 \),
\( Y_3 \), what are the order statistics of the outcome?

\hr

\visual{Key Concepts}{../../../../CommonInformation/Lessons/keyconcepts.png}
\section*{Key Concepts}

\begin{enumerate}
    \item
        \emph{Risk} is a board game of battles between two players, and
        each battle consists of one or more rounds of combat between \(
        A \) attacking armies and \( D \) defending armies.  The goal is
        to develop the probability of winning the battle as a function
        of \( A \) and \( D \) using a Markov chain with \( AD + A + D \)
        states.
\end{enumerate}

\hr

\visual{Vocabulary}{../../../../CommonInformation/Lessons/vocabulary.png}
\section*{Vocabulary}
\begin{enumerate}
    \item
        Given a sample of \( n \) identically distributed random values,
        the \( \ell \)th \defn{order statistic} is the \( \ell \)th
        largest value.%
        \index{order statistics}
\end{enumerate}

\hr

\section*{Notation}
\begin{enumerate}
    \item
        \( A \) -- the number of attacking armies
    \item
        \( D \) -- the number of defending armies
    \item
        \( X_n \) -- the state of the system at the beginning of the \(
        n \)th turn:
        \[
            X_n = (a_n, d_n), 0 \le a_n \le A, 0 \le d_n \le D
        \] where \( a_n \) and \( d_n \) are the number of attacking and
        defending armies respectively
    \item
        \( \rho_{i j \ell} \) -- the joint probabilities associated with
        the best and second best roll from \( 2 \) or \( 3 \) six-sided
        fair dice
    \item
        \( Y_1 \), \( Y_2 \), \( Y_3 \) -- the unordered outcome for an
        attacker when rolling three dice
    \item
        \( W_1 \), \( W_2 \) -- the unordered outcome for an attacker
        when rolling two dice
    \item
        \( Z_1 \), \( Z_2 \) -- the unordered outcome for a defender
        rolling two dice
    \item
        \( Y^{(1)} \ge Y^{(2)} \ge Y^{(3)} \) -- the ordered random
        variables are the \emph{order statistics}
    \item
        \( f_{ij}^{(n)} \) the probability that the first (and final)
        visit form state \( i \) to absorbing state \( j \) occurs on
        the \( n \)th turn
    \item
        \( F^{(n)} \) -- \( AD \times (A + D) \) matrix of these first
        transitions
    \item
        \( F = (I - Q)^{-1} R \) -- probability that the system
        eventually goes from transient state \( i \) to absorbing state \(
        j \)
    \item
        \( R_A \) and \( R_D \) -- the number of armies remaining for
        the attacker and defender respectively given the initial state \(
        X_0 = (A, D) \) probabilities
\end{enumerate}

\visual{Mathematical Ideas}{../../../../CommonInformation/Lessons/mathematicalideas.png}
\section*{Mathematical Ideas}

This section uses Markov chains to analyze the stochastic progress of a
battle between two players in the board game \emph{Risk}.  Standard
Markov chain methods give the probabilities of victory and additionally
the expected losses in battle.

\subsection*{The Game of \emph{Risk}}

The board game \emph{Risk} for two or more players has a world map
separated into \( 42 \) territories.  Each player has a number of tokens
used as armies.  Each token represents one army.  A player may use some
or all but one of the armies to attack the armies in an adjacent
opponent-controlled territory.  The goal of a player is either to
conquer the world through battles or to fulfill a task assigned at the
beginning of the game.  This task can be destroying a specific enemy or
conquering a specific area.%
\index{\emph{Risk}}

The game consists of battles between two players, and each battle
consists of one or more rounds of combat.  At each turn, a player must
decide whether or not to attack a territory.  The first question is the
following:  \emph{If a player attacks a territory with your armies, what
is the probability the player will capture this territory?} Of course,
the probability that a player will capture a territory could be high
while the expected loss is also high.  Therefore, the second question
is: \emph{If a player engages in a war, how many armies should the
player expect to keep, depending on the number of armies the opponent
has on that territory?}

At each turn, a player occupying some region may attack a neighboring
territory.  An attacker must leave at least one army on the occupied
territory, and use at least one army in an attack.  Thus a player who
has at least two armies (one to keep on the territory and one to attack
a territory) may attack a neighbor territory.  A player who decides to
attack a territory declares war on the opponent and the two players
engage in a sequence of combats with random outcomes determined by dice.
Outcomes of the defender's and the attacker's dice decide each battle
and the ultimate outcome of the war.  An attacker may withdraw before
the war ends.  An attacker who does not withdraw either destroys all the
defender armies in that territory and occupies that territory, or loses
all the attacking armies and fails to conquer the territory.

The number of armies an attacker or a defender possesses at the start of
a each combat determines the number of dice each player rolls.  To begin
a round of battle, the attacker will roll one die if attacking with a
single army, two dice if the attacker has two armies present, and three
dice if the attacking player has three or more armies involved in the
battle.  A defender rolls one die with one army and two dice with two or
more armies in the embattled territory.  Six different cases result from
the number of dice each opponent rolls, see Table~%
\ref{tab:riskgame:riskdice}.

\begin{table}
    \centering
    \begin{tabular}{ccccc}
             & \multicolumn{2}{c}{Number of Armies} & \multicolumn{2}{c}{Number of Dice} \\ 
        Case & Attacker & Defender & Attacker & Defender \\ 
        1    & $1$      & $1$      & $1$      & $1$      \\ 
        2    & $2$      & $1$      & $2$      & $1$      \\ 
        3    & $ \ge 3$ & $1$      & $3$      & $1$      \\ 
        4    & $1$      & $ \ge 2$ & $1$      & $2$      \\ 
        5    & $2$      & $ \ge 2$ & $2$      & $2$      \\ 
        6    & $ \ge 3$ & $ \ge 2$ & $3$      & $2$      \\ 
    \end{tabular}
    \caption{The number of dice each player rolls according to the
    number of attacker and defender armies.}%
    \label{tab:riskgame:riskdice}
\end{table}

At each turn, after rolling the dice, each player sets the dice in
descending order and then pairs off the two sets of dice, matching the
highest attacking die roll against the highest defending die roll.  If
each player is using at least two dice, then the second highest die
rolls are similarly matched.  The attacker loses one army for each die
that is less than or equal to the corresponding defender's die.  The
defender loses one army for each die that is strictly less than the
corresponding attacker's die.  Note that ties favor the defender, so
each tie results in the loss of an attacking army.  After taking away
the armies lost at that turn, the players roll the dice again.  This
sequence continues until one side loses all of its armies.  Table~%
\ref{tab:riskgame:riskwar} shows the progress of a typical war for a
country.  This war took four rounds.  At the end of the fourth turn, the
attacker lost all of the attacking armies and the defender wins the war.
This table also serves as a reminder of the number of dice rolled in
various situations, never more than three for the attacker, and never
more than two for the defender.
\begin{table}
    \centering
    \begin{tabular}{ccccccccc}
        Roll & \multicolumn{2}{c}{Armies} & \multicolumn{2}{c}{Dice Rolled} & \multicolumn{2}{c}{Dice Outcome} & \multicolumn{2}{c}{Losses} \\ 
             & Att      & Def      & Att      & Def      & Att   & Def & Att & Def \\ 
        1    & 4        & 3        & 3        & 2        & 5,4,3 & 6,3 & 1   & 1   \\ 
        2    & 3        & 2        & 3        & 2        & 5,5,3 & 5,5 & 2   & 0   \\ 
        3    & 1        & 2        & 1        & 2        & 6     & 4,3 & 0   & 1   \\ 
        4    & 1        & 1        & 1        & 1        & 5     & 6   & 1   & 0   \\ 
        5    & 0        & 1        &          &          &       &     &     &     \\ 
    \end{tabular}
    \caption{An example of a battle in \emph{Risk}}.%
    \label{tab:riskgame:riskwar}
\end{table}

It may be disadvantageous to take control of a new territory by winning
a war with only one army remaining, emphasizing the importance of the
second question:  \emph{If a player engages in a war, how many armies
should the player expect to lose depending on the number of armies the
opponent has on that territory?}

\subsection*{\emph{Risk} States}

Let \( A \) be the number of attacking armies and \( D \) the number of
defending armies.  The values of \( A \) and \( D \) characterize the
state of the system at a given time.  Let \( X_n \) be the state of the
system at the beginning of the \( n \)th turn:
\[
    X_n = (a_n, d_n), 0 \le a_n \le A, 0 \le d_n \le D
\] where \( a_n \) and \( d_n \) are the number of attacking and
defending armies respectively.  The initial state of the system is \( X_0
= (A, D) \).  If one side loses all its armies then that side loses the
war; i.e., if \( X_n= (0, d_n) \) with \( d_n > 0 \), then the attacker
has lost the war at the end of the turn.  Similarly, if \( X_n= (a_n ,0)
\), with \( a_n > 0 \), then the attacker has won the war at the end of
the turn.  Given the number of armies each side has at a given round,
then it is possible to calculate the probability that each side wins or
loses the war without knowing the states of the system before that
round.  In other words, the battle process has the Markov property.

The \( AD \) states where both \( a \) and \( d \) are positive are
transient.  The \( A + D \) states where either \( a = 0 \) or \( d = 0 \)
are absorbing.  Order the \( AD + A + D \) possible states so that the \(
AD \) transient states precede the \( A + D \) absorbing states. Order
the transient states with the attacker armies decreasing along a row for
constant defenders, and the defender armies decreasing row by row, see
Figure~%
\ref{fig:riskgame:risktrans}.  The transient states are, in order
\begin{align*}
    & (A, D), (A-1, D), (A-2, D), \dots, (1, D), \\
        & (A, D-1), (A-1, D-1), \dots, (1, D-1), \\
        & \dots \\
        & (A, 1), (A-1, 1), \dots, (1,1)
\end{align*}
and order the absorbing states by decreasing desirability of outcome for
the attacker, including the absorbing states where the defender wins
\[
    (A, 0), (A-1, 0),\dots, (1, 0), (0, 1), (0, 2),\dots, (0, D).
\] Note that this order is the reverse of the state order in
\cite{osborne03}.  The order here proceeds in the general direction of
the battle, with the first state for the initial numbers of armies, then
decreasing number of armies for both players.  See Figure~%
\ref{fig:riskgame:risktrans} for a diagram of the possible state
transitions when \( A = 5 \) and \( D = 4 \).  This figure is key for
understanding the Markov chain.  The small number below each ordered
pair state is the index of the state for the transition probability
matrix.  The state indices increase from left to right and top to
bottom, so the state of the system flows from the upper left corner to
the bottom or right margin.  The diagram divides into \( 6 \) regions, \(
3 \) for the attacker, and \( 2 \) for the defender depending on the
number of dice each is rolling in a battle, as already outlined in
Table~%
\ref{tab:riskgame:riskdice}.  For example, the lower right region
corresponds to Case 1 in Table~%
\ref{tab:riskgame:riskdice} and the transition probabilities \( \rho_{111}
\) and \( \rho_{110} \) detailed below in Table~%
\ref{tab:riskgame:risktransprobs}.  The largest region in the upper left
corresponds to Case 6 in Table~%
\ref{tab:riskgame:riskdice} and the transition probabilities \( \rho_{32\ell}
\) in Table~%
\ref{tab:riskgame:risktransprobs}.  Likewise for the other regions.
Figure~%
\ref{fig:riskgame:risktrans} labels the possible state transitions with
arrows and colors corresponding to the transition probabilities in
Table~%
\ref{tab:riskgame:risktransprobs}.

The states and the coordinates of states on the diagram can be
confusing.  For instance, the initial armies state \( (5,4) \) is at
location \( (x=1, y=4) \) on the diagram.  As the attacking armies
decrease, the first coordinate of the location of the state increases.
As the defending armies decrease, the second coordinate of the location
decreases.  This means first that writing computer code for the chart
requires close attention to indices and second to not confuse location
with state.

\begin{figure}
    \centering
\begin{asy}
settings.outformat = "pdf";

// import graph;

size(5inches);

real myfontsize = 12;
real mylineskip = 1.2*myfontsize;
pen mypen = fontsize(myfontsize, mylineskip);
defaultpen(mypen);

draw( (-1, 1.5)--(6.5, 1.5), dashed);
label( "Defender dice:", (-0.5, 4));
label( "Two", (-0.5, 1.5), N);
label( "One", (-0.5, 1.5), S);

draw( (3.5, -0.75)--(3.5, 4.5), dashed);
draw( (4.5, -0.75)--(4.5, 4.5), dashed);
label( "Attacker dice:", (1.0, -0.5));
label( "One", (4.5, -0.5), E);
label( "Two", (4.0, -0.5));
label( "Three", (3.5, -0.5), W);

int index = 0;
for (int j=4; j > 0; --j) {	// top to bottom
  for (int i=1; i <= 5; ++i) {	// left to right
    label("("+format("%d", 6-i)+","+format("%d", j)+")", (i,j) );
    label(scale(0.5)*format("%d", ++index), (i-0.2, j-0.2));
  }
}
for (int i=1; i <= 5; ++i) {
  label("("+format("%d", 6-i)+",0)", (i,0) );
    label(scale(0.5)*format("%d", ++index), (i-0.2, 0-0.2));
}
for (int j=1; j <= 4; ++j) {
  label("(0,"+format("%d", j)+")", (6,j) );
    label(scale(0.5)*format("%d", ++index), (6-0.2, j-0.2));
}

real eps = 0.2;

for (int j=2; j <= 4; ++j) {
  for (int i=1; i <= 3; ++i) {
    draw(((i,j)+ eps*SE)--((i+1, j-1)+eps*NW), blue, Arrow);
    draw(( (i,j)+ eps*SW){SSW}..((i, j-2)+eps*NW){SSE}, green, Arrow);
    draw(( (i,j)+ eps*NE){ENE}..((i+2, j)+eps*NW){ESE}, red, Arrow);
  }
}

for (int j=2; j <= 4; ++j) {
  draw(((4,j)+ eps*NE){ENE}..((6, j)+eps*NW){ESE}, cyan, Arrow);
  draw(((4,j)+ eps*SW){SSW}..((4, j-2)+eps*NW){SSE}, yellow, Arrow);
  draw(((4,j)+ eps*SE)--((5, j-1)+eps*NW), darkgreen, Arrow);
  draw(((5,j)+ eps*S)--((5, j-1)+eps*N), brown, Arrow);
  draw(((5,j)+ 1.5*eps*E)--((6, j)+ 1.5*eps*W){SSE}, purple, Arrow);
}

for (int i=1; i <= 3; ++i) {
  draw(((i,1)+ 1.5*eps*E)--((i+1, 1)+ 1.5*eps*W), gray, Arrow);
  draw(((i,1)+ eps*S)--((i, 0)+eps*N), orange, Arrow);
}

draw(((4,1)+ 1.5*eps*S)--((4, 0)+ 1.5*eps*N), pink, Arrow);
draw(((5,1)+ eps*S)--((5, 0)+eps*N), lightgray, Arrow);

draw(((4,1)+ 1.5*eps*E)--((5, 1) + 1.5*eps*W), black, Arrow);
draw(((5,1)+ 1.5*eps*E)--((6, 1) + 1.5*eps*W), magenta, Arrow);
\end{asy}
    \caption{The state space and transitions of the Markov chain for a
    war in \emph{Risk}.  Transitions of the same color have the same
    probability.}%
    \label{fig:riskgame:risktrans}
\end{figure}

\subsection*{Transition Probabilities}

Under this ordering, the transition probability matrix takes the
block-matrix form
\[
    P =
    \begin{pmatrix}
        Q       & R \\
        0       & I
    \end{pmatrix}
\] where the \( (AD) \times (AD) \) matrix \( Q \) has the probabilities
of going from one transient state to another, and the \( (AD) \times (A
+ D) \) matrix \( R \) has the probabilities of going from a transient
state into an absorbing state.  \( P \) contains only \( 14 \) distinct
probability values, based on the number of dice rolled and the armies
lost as a result.  Let \( \rho_{i j \ell} \) denote the probability that
the defender loses \( \ell \) armies when rolling \( j \) dice against
an attacker rolling \( i \) dice, as given in Table~%
\ref{tab:riskgame:risktransprobs}.  To calculate the \( \rho_{i j \ell} \),
find the joint probabilities associated with the best and second best
roll from \( 2 \) or \( 3 \) six-sided fair dice.  Let \( Y_1 \), \( Y_2
\), \( Y_3 \) denote the unordered outcome for an attacker when rolling
three dice and let \( W_1 \), \( W_2 \) denote the unordered outcome for
an attacker when rolling two dice.  Let \( Z_1 \), \( Z_2 \) denote the
unordered outcome for a defender rolling two dice.  Then \( Y_1 \), \( Y_2
\), \( Y_3 \), \( W_1 \), \( W_2 \) and \( Z_1 \), \( Z_2 \) are random
samples from the discrete uniform distribution on the integers \( 1 \)
through \( 6 \).  For example,
\[
    \Prob{Y_j = y} =
    \begin{cases}
        \frac{1}{6}     & y = 1,2,3,4,5,6 \\
        0       & \text{else}.
    \end{cases}
\]

Given a sample of \( n \) identically distributed random values, the \(
\ell \)th \defn{order statistic} is the \( \ell \)th largest value.%
\index{order statistics}
Using superscripts, the ordered random variables \( Y^{(1)} \ge Y^{(2)}
\ge Y^{(3)} \) are the \emph{order statistics} for the dice roll.  The
joint distributions of the order statistics for specification of \( \rho_
{i j \ell} \) use straightforward techniques of enumeration.  When
rolling two dice, the joint distribution of \( (Z^{(1)} , Z^{(2)}) \) is
\[
    \Prob{Z^{(1)} = z^{(1)}, Z^{(2)} = z^{(2)}} =
    \begin{cases}
        \frac{1}{36}    & z^{(1)} = z^{(2)}\\
        \frac{2}{36}    & z^{(1)} > z^{(2)}\\
        0       & \text{else}.
    \end{cases}
\] See Figure~%
\ref{fig:riskgame:ordertwo} for a visualization of how to obtain this
probability for \( Z^{(1)} = 5, Z^{(2)} = 3 \).
\begin{figure}
    \centering
\begin{asy}
      settings.outformat = "pdf";

      import math;
      size(2.5inches,0,IgnoreAspect);

      real myfontsize = 12;
      real mylineskip = 1.2*myfontsize;
      pen mypen = fontsize(myfontsize, mylineskip);
      defaultpen(mypen);

      void tableprobs(int m=6, int n=6){
          string[][] y= new string[m+1][n+1];
          for (int i=1; i<=m; ++i)
          {
              for (int j=1; j<=n; ++j)
              {
                y[i][j]="$\frac{1}{36}$";
                if (i==5 && j==3) {
                  label(y[i][j],(j,-i), red);
                } else {
                  if (i==3 && j==5) {
                  label(y[i][j],(j,-i), red);
                  } else {
                if(i==1) label(format("%i",j),(j,0),blue);
                }
              }
              label(format("%i",i),(0,-i),blue);
          }
          add(shift(-.5,-.5-m)*grid(n+1,m+1,blue));
        }
      }

      tableprobs();

      shipout(bbox(.25cm,Fill(white)));
\end{asy}
    \caption{Visualization of \( Z^{(1)} = 5 \), \( Z^{(2)} = 3 \).}%
    \label{fig:riskgame:ordertwo}
\end{figure}
The \emph{marginal distribution} of the best roll \( Z^{(1)} \) sums the
joint distribution over values of \( z^{(2)} \):
\[
    \Prob{Z^{(1)} = z^{(1)}} = \frac{2z^{(1)} - 1}{36}
\] for \( z^{(1)} = 1,2,3,4,5,6 \).  See Figure~%
\ref{fig:riskgame:orderone} for a visualization of how to obtain this
probability for \( Z^{(1)} = 5 \).
\begin{figure}
    \centering
\begin{asy}
    settings.outformat = "pdf";

    import math;
    size(2.5inches,0,IgnoreAspect);

    real myfontsize = 12;
    real mylineskip = 1.2*myfontsize;
    pen mypen = fontsize(myfontsize, mylineskip);
    defaultpen(mypen);

    void tableprobs(int m=6, int n=6){
      string[][] y= new string[m+1][n+1];
      for (int i=1; i<=m; ++i) {
        for (int j=1; j<=n; ++j) {
          y[i][j]="\(\frac{1}{36} \)";
          if (i==5 || j==5) {
            label(y[i][j],(j,-i), red);
        }
        else {
          label(y[i][j],(j,-i), black);
        }
        if(i==1) label(format("%i",j),(j,0),blue);
      } label(format("%i",i),(0,-i),blue);
    }
    add(shift(-.5,-.5-m)*grid(n+1,m+1,blue)); }

    tableprobs();

    shipout(bbox(.25cm,Fill(white)));
\end{asy}
    \caption{Visualization of \( Z^{(1)} = 5 \).}%
    \label{fig:riskgame:orderone}
\end{figure}
When rolling three dice, the distribution of the best two rolls is
\[
    \Prob{Y^{(1)} = y^{(1)}, Y^{(2)} = y^{(2)}} =
    \begin{cases}
        \frac{3y^{(1)} - 2}{216}        & y^{(1)} = y^{(2)}\\
        \frac{6y^{(2)}-3}{36}   & y^{(1)} > y^{(2)}\\
        0       & \text{else}.
    \end{cases}
\] See Figure~%
\ref{fig:riskgame:threeordertwo} for a visualization of how to obtain
this probability for \( Y^{(1)} = 3 \), \( Y^{(2)} = 3 \).  See the
exercises for the derivation of the entry for \( Y^{(1)} = y^{(1)} \), \(
Y^{(2)} = y^{(2)} \) with \( y^{(1)} > y^{(2)} \).
\begin{figure}
    \centering
\begin{asy}
    settings.outformat="pdf";
    settings.prc=false;
    settings.render=0;

    import grid3; 
    size3(150,IgnoreAspect);

    real myfontsize = 12;
    real mylineskip = 1.2*myfontsize;
    pen mypen = fontsize(myfontsize, mylineskip);
    defaultpen(mypen);


    limits((0,0,0),(6,6,6));
    currentprojection=perspective(camera=(20,16,7),target=(5,5,3));

    draw(box((2,2,0),(3,3,3)),red);
    draw(box((2,0,2),(3,3,3)),red);
    draw(box((0,2,2),(3,3,3)),red);

    grid3(XYZgrid,Step=1);

    xaxis3(Label("$y^{(1)}$",MidPoint,align=Y-Z),
           Bounds(Both,Min),OutTicks(beginlabel=false),p=blue); 
    yaxis3(Label("$y^{(2)}$",MidPoint,align=X-Z),
           Bounds(Both,Min),OutTicks(beginlabel=false),p=blue); 
    zaxis3(Label("$y^{(3)}$",MidPoint,align=X-Y),
           Bounds(Both,Min),InTicks(beginlabel=false),p=1bp+blue);
\end{asy}
    \caption{Visualization of \( Y^{(1)} = 3, Y^{(2)} = 3 \).}%
    \label{fig:riskgame:threeordertwo}
\end{figure}
The marginal distribution of the best roll is
\[
    \Prob{Y^{(1)} = y^{(1)}} = \frac{1 - 3y^{(1)} + 3 \left( y^{(1)}
    \right)^2}{216}
\] for \( y^{(1)} = 1,2,3,4,5,6 \).  All of the probabilities are \( 0 \)
for arguments that are not positive integers less than or equal to \( 6 \).
The joint distribution of \( W^{(1)} \) and \( W^{(2)} \) is the same as
that for \( Z^{(1)} \) and \( Z^{(2)} \).

Now it is possible to calculate specific transition probabilities using
these marginal distributions.  For example
\begin{align*}
    \rho_{322}  &= \Prob{Y^{(1)} > Z^{(1)}, Y^{(2)} > Z^{(2)}} \\
        &= \sum\limits_{z_1=1}^5 \sum\limits_{z_2=1}^{z_1} \Prob{Y^{(1)}
    > z_1, Y^{(2)} > z_2} \Prob{Z^{(1)} = z_1, Z^{(2)} = z_2} \\
        &= \sum\limits_{z_1=1}^5 \sum\limits_{z_2=1}^{z_1} \sum\limits_{y_1
    = z_1+1}^6 \sum\limits_{y_2 = z_2+1}^{y_1} \Prob{Y^{(1)} = y_1, Y^{(2)}
    = y_2} \Prob{Z^{(1)} = z_1, Z^{(2)} = z_2} \\
        &= \frac{2890}{7776} \\
        &\approx 0.372
\end{align*}
Note that those events in this quadruple sum for which an argument with
a subscript of 2 exceeds an argument with the same letter and subscript
1 have probability zero.

Obtain the probabilities \( \rho \) of the transition probability matrix
\( P \) similarly using the joint distributions for \( Y^{(1)} \), \( Y^
{(2)} \), for \( Z^{(1)} \), \( Z^{(2)} \), and for \( W^{(1)} \), \( W^
{(2)} \).  The exact probabilities, rounded to \( 3 \) decimal places,
are in Table~%
\ref{tab:riskgame:risktransprobs}.  The colors refer to the colors of
the arrows representing transitions in Figure~%
\ref{fig:riskgame:risktrans}.

\begin{table}
    \centering
    \begin{tabular}{cclcrrl}
        Att    & Def      & Outcome     & Symbol       & Exact       & Approx. & Color      \\ 
        Dice   & Dice     &             &              & Prob.       & Prob.   &            \\ 
        \hline
        1      & 1        & Def loses 1 & $\rho_{111}$ & $15/36$     & $0.417$ & light gray \\ % checked
         
        1      & 1        & Att loses 1 & $\rho_{110}$ & $21/36$     & $0.583$ & magenta    \\ % checked
         
        \hline
        1      & 2        & Def loses 1 & $\rho_{121}$ & $55/216$    & $0.255$ & brown      \\ % checked   
         
        1      & 2        & Att loses 1 & $\rho_{120}$ & $161/216$   & $0.745$ & purple     \\ % checked
         
        \hline
        2      & 1        & Def loses 1 & $\rho_{211}$ & $125/216$   & $0.579$ & pink       \\ % checked
         
        2      & 1        & Att loses 1 & $\rho_{210}$ & $91/216$    & $0.421$ & black      \\ % checked
         
        \hline
        2      & 2        & Def loses 2 & $\rho_{222}$ & $295/1296$  & $0.228$  & cyan       \\ % checked
         
        2      & 2        & Each lose 1 & $\rho_{221}$ & $420/1296$  & $0.324$ & darkgreen  \\ 
        2      & 2        & Att loses 2 & $\rho_{220}$ & $581/1296$  & $0.448$ & yellow     \\ %checked
         
        \hline
        3      & 1        & Def loses 1 & $\rho_{311}$ & $855/1296$  & $0.660$ & orange     \\ % checked
         
        3      & 1        & Att loses 1 & $\rho_{310}$ & $441/1296$  & $0.340$ & gray       \\ % checked
         
        \hline
        3      & 2        & Def loses 2 & $\rho_{322}$ & $2890/7776$ & $0.372$ & green      \\ % checked
         
        3      & 2        & Each lose 1 & $\rho_{321}$ & $2611/7776$ & $0.336$ & blue       \\ 
        3      & 2        & Att loses 2 & $\rho_{320}$ & $2275/7776$ & $0.293$ & red        \\ % checked
         
    \end{tabular}
    \caption{Probabilities in the transition probability matrix.  The
    colors correspond to the arrows in the diagram of state transitions.}
    \label{tab:riskgame:risktransprobs}
\end{table}

\subsection*{Absorption Probabilities}

Given any initial state, the system will eventually make a transition to
an absorbing state.  For a transient state \( i \), call \( f_{ij}^{(n)}
\) the probability that the first (and final) visit to absorbing state \(
j \) occurs on the \( n \)th turn:
\[
    f_{ij}^{(n)} = \Prob{X_n = j, X_k \ne j \text{ for } k=1,2,\dots,
    n-1 \given X_0 = i}.
\] Denote the \( AD \times (A + D) \) matrix of these first transition
probabilities by \( F^{(n)} \).  Then \( F^{(n)} = Q^{n-1}R \).  The
probability that the system eventually goes from transient state \( i \)
to absorbing state \( j \) is \( F = (I - Q)^{-1} R \).~%
\index{fundamental matrix}
If the system ends in one of the \( A \) absorbing states then the
attacker wins; if it ends in one of the \( D \) absorbing states, the
defender wins.  Since the initial state of a battle is the first state
using the order established previously, the probability that the
attacker wins is just the sum of the entries in the first row of the
submatrix of the first \( A \) columns of \( F \):
\[
    \Prob{\text{Attacker wins} \given X_0 = (A,D)} = \sum\limits_{j=1}^{A}
    f_{1, j}
\] and
\[
    \Prob{\text{Defender wins} \given X_0 = (A,D)} = \sum\limits_{j=A+1}^
    {A+D} f_{1, j}.
\]

\begin{small}
    \begin{table}
        \centering
        \begin{tabular}{lrrrrrrrrrr}
            Att    & 10       & 9           & 8            & 7           & 6       & 5          & 4     & 3     & 2     & 1     \\ 
            Def    &          &             &              &             &         &            &       &       &       &       \\ 
            10     & 0.568    & 0.480       & 0.380        & 0.287       & 0.193   & 0.118      & 0.057 & 0.021 & 0.001 & 0.000 \\ 
            9      & 0.650    & 0.558       & 0.464        & 0.357       & 0.258   & 0.162      & 0.086 & 0.033 & 0.003 & 0.000 \\ 
            8      & 0.724    & 0.646       & 0.547        & 0.446       & 0.329   & 0.224      & 0.123 & 0.054 & 0.005 & 0.000 \\ 
            7      & 0.800    & 0.726       & 0.643        & 0.536       & 0.423   & 0.297      & 0.181 & 0.084 & 0.011 & 0.000 \\ 
            6      & 0.861    & 0.808       & 0.730        & 0.640       & 0.521   & 0.397      & 0.253 & 0.134 & 0.021 & 0.000 \\ 
            5      & 0.916    & 0.873       & 0.818        & 0.736       & 0.638   & 0.506      & 0.359 & 0.206 & 0.049 & 0.002 \\ 
            4      & 0.954    & 0.930       & 0.888        & 0.834       & 0.745   & 0.638      & 0.477 & 0.315 & 0.091 & 0.007 \\ 
            3      & 0.981    & 0.967       & 0.947        & 0.910       & 0.857   & 0.769      & 0.642 & 0.470 & 0.206 & 0.027 \\ 
            2      & 0.994    & 0.990       & 0.980        & 0.967       & 0.934   & 0.890      & 0.785 & 0.656 & 0.363 & 0.106 \\ 
            1      & 1.000    & 1.000       & 1.000        & 0.999       & 0.997   & 0.990      & 0.972 & 0.916 & 0.754 & 0.417 \\ 
        \end{tabular}
        \caption{Probability that the attacker wins with \( A \) armies
        against \( D \) defending armies.}%
        \label{tab:riskgame:attackerwins}
    \end{table}
\end{small}

Because the state order here is the reverse of the order given in
\cite{osborne03}, the table of winning probabilities here is the reverse
of Table 3 in that paper.  Note also by the Markov property, by
calculating the winning probability for say \( A = 10 \) and \( D=10 \),
the winning probability for any war starting with \( A < 10 \) and \( D
< 10 \) is also automatically calculated so that the entire Table~%
\ref{tab:riskgame:attackerwins} comes from \( N = (I-Q)^{-1} \) for \( A
= 10 \) and \( D = 10 \).

Large engagements occur towards the end of an entire game of \emph{Risk}.
The absorbing probabilities for large battles have an interesting
``saw-tooth'' feature noticed in
\cite{pierce15}.  Figure~%
\ref{fig:riskgame:finalstateprob} shows the probability mass function
when the attacker and defender each start with \( 40 \) armies.  The
horizontal axis shows all the absorbing states of a \( (40,40) \) battle
and the vertical axis is the probability of the corresponding absorbing
state.  The outcome states along the horizontal axis are in decreasing
order of preference of the attacker.  One distinctive feature of this
probability mass function is the local mimimum at \( 40 \) in the
probability masses. The outcomes \( (1,0) \) and \( (0,1) \) are less
likely than their neighbors \( (2,0) \) and \( (0,2) \).  The outcomes \(
(2,0) \) and \( (0,2) \) can occur when a player with a single army will
roll only one die against two dice and is less likely to survive.  Also
interesting is the jagged or ``saw-tooth'' phenomenon near the local
maxima.  A final outcome of \( (8,0) \) or \( (6,0) \) is each more
likely than an outcome of \( (7,0) \).  In the largest part of the
transient states the probability of removing \emph{two} armies from one
player is about double the probability of each player losing \emph{one}
army.  So a path that begins at \( (40,40) \) is more likely to visit
only states \( (a,d) \) in which \( a \) and \( d \) have the same
parity, except when one player keeps just one army.

\begin{figure}
    \centering
    \includegraphics[scale=0.33]{finalStateProb}
    \caption{The probability mass function of absorbing states for \( A
    = 40 \), \( D = 40 \).  The absorbing states are in decreasing order
    of preference for the attacker.}%
    \label{fig:riskgame:finalstateprob}
\end{figure}

\subsection*{Expected Losses}

Using the matrix \( NR \) allows calculation of the expected values and
variances for the losses that the attacker and defender will suffer in a
battle.  For example, suppose \( A = 5 \) and \( D = 4 \), as
illustrated in Figure~%
\ref{fig:riskgame:risktrans}.  In this case, the first row of the \( 20
\times 9 \) matrix \( NR \) gives the probabilities for the \( A + D = 9
\) absorbing states:  \( (NR)_{1,\cdot} = 0.138, 0.165, 0.179, 0.107,
0.050, 0.070, 0.124, 0.104, 0.064 \).  Let \( R_A \) and \( R_D \)
denote the number of armies remaining for the attacker and defender
respectively given the initial state \( X_0 = (A, D) \).  The
probability distributions for \( R_A \) and \( R_D \) are in the first
row of \( NR \):
\[
    \Prob{R_A = k} =
    \begin{cases}
        \sum\limits_{\nu=A+1}^{A+D} (NR)_{1,\nu}        & k = 0, \\
        (NR)_{1,A+1-k}  & k = 1, 2, \dots A
    \end{cases}
\] and
\[
    \Prob{R_D = k} =
    \begin{cases}
        \sum\limits_{\nu=1}^{A} (NR)_{1,\nu}    & k=0, \\
        (NR)_{1,k}      & k = A+1, A+2, \dots A+D.
    \end{cases}
\] For \( A = 5 \) and \( D = 4 \) the mean and standard deviation for
the attacker's remaining armies are \( \E{R_A} = 2.15 \) and \( \sqrt{\Var
{R_A}} = 1.89 \).  The defender's mean and standard deviation in this
case are \( \E{R_D} = 0.886 \) and \( \sqrt{\Var{R_D}} = 1.32 \).  The
attacker has an advantage in the sense that expected losses, \( 2.85 \),
are lower than for the defender, \( 3.1 \).  This is generally rue
provided the initial number of attacking armies is not too small.

To get an overall sense of expected remaining armies, Figure~%
\ref{fig:riskgame:expectedrem} has heatplots of expected losses for
values of \( A \) and \( D \) between \( 1 \) and \( 20 \).  A heatmap
is a graphical representation of data where the individual values
contained in a matrix are represented as colors.  Here the lighter
colors represent larger numbers of armies remaining, the darker colors
smaller number numers of armies.  The matrix is the number of armies
remaining after a battle between a number of attackers along a column
and defenders along a row.  The scripts below calculate the expected
number of armies remaining for the numerical values.
\begin{figure}
    \centering
    \begin{minipage}[b]
        {0.40\linewidth} \includegraphics[scale=0.30]{attackersRemain.png}
    \end{minipage}
    \begin{minipage}[b]
        {0.40\linewidth} \includegraphics[scale=0.30]{defendersRemain.png}
    \end{minipage}
    \caption{Expected losses for values of \( A \) on the left and \( D \)
    on the right.  Lighter colors represent larger numbers of armies
    remaining, the darker colors smaller number numers of armies.  Note
    the axis scales, the number of attackers decrease left to right, the
    number of defenders decrease top to bottom, as in Figure~%
    \ref{fig:riskgame:risktrans}}~%
    \label{fig:riskgame:expectedrem}
\end{figure}

When the number of attacking and defending armies is equal, the
probability that the attacker ends up winning the territory exceeds \(
50\% \), provided the initial stakes are high enough, at least \( 5 \)
armies each, initially, see Table~%
\ref{tab:riskgame:equalwins}, With \( 40 \) armies attacking \( 40 \)
defending armies, the expected armies remaining are \( 7.5 \) attacking
armies and \( 2.1 \) defending armies.  The overall conclusion is that
the chances of winning a battle are more favorable for the attacker.
The logical recommendation is for the attacker to be aggressive.

\begin{table}
    \centering
    \begin{tabular}{rrrrrrrrrr}
        20     & 19       & 18          & 17           & 16          & 15      & 14         & 13    & 12    & 11    \\ 
        0.633  & 0.628    & 0.623       & 0.617        & 0.611       & 0.605   & 0.599      & 0.592 & 0.584 & 0.576 \\ 
        10     & 9        & 8           & 7            & 6           & 5       & 4          & 3     & 2     & 1     \\ 
        0.568  & 0.558    & 0.547       & 0.536        & 0.521       & 0.506   & 0.477      & 0.470 & 0.363 & 0.417
    \end{tabular}
    \caption{Probability of attacker winning when the number of armies
    are equal.}%
    \label{tab:riskgame:equalwins}
\end{table}

Using the absorption probabilities for the states \( (a, 0) \) where the
attacker wins allows a more detailed summary of the outcomes of the war.
As an example, suppose the attacker wants at least a probability of \(
0.75 \) of winning the war with at least four armies surviving.  With
how many armies should the attacker start a war to create this favorable
situation?  Refer to Figure~%
\ref{fig:riskgame:quartiles} for an example of how to answer this
important strategic question.  Each facet in the figure indicates the
number of defending armies from \( 1 \) to \( 4 \).  The horizontal axis
in each facet lists the number of starting attacking armies, while the
vertical axis in each facet displays how many attacking armies remain
after the battle.  Respectively, the blue, green and red curves with
points indicate the first-quartile, median, and third-quartile states in
the distribution of outcomes based on the number of the attacker's
starting armies.  The outcomes are in descending order of preference for
the attacker.  For example, consider the lower right facet corresponding
to the defender starting with \( 4 \) armies.  If the attacker has seven
armies, the most desired outcome is the absorbing state \( (7,0) \),
while the least desired is \( (0,4) \).  In this case, the median
outcome is \( (4,0) \) in the sense that \( (a,0) \) with \( a \ge 4 \)
has a probability of at least \( 0.50 \) (in fact, \( \Prob{a \ge 4, d =
0} = 0.622 \).) This means that with probability at least \( 0.50 \),
the attacker will finish the war with \( 4 \) or more armies.  The
interquartile range is \( (6,0) \) through \( (2,0) \).  This means that
with probability at least \( 0.25 \) (precisely, the probability is \(
0.303 \)) the attacker will win the war and have \( 6 \) or more armies
remaining; with probability at least \( 0.50 \) (precisely, the
probability is \( 0.503 \)) the attacker will win and have \( 2 \) to \(
6 \) armies remaining; and the remaining \( 0.25 \) probability (precisely,
the probability is \( 0.195 \)) of battles will leave the attacker with \(
2 \) or fewer armies.  Returning to the question of how many armies the
player needs to start with to attack a defender with \( 4 \) armies with
at least a probability of \( 0.75 \) of winning with at least four
armies remaining, the attacker should start with at least \( 9 \)
armies.  In this case, the interquartile range is \( (8,0) \) through \(
(4,0) \) and the median outcome is \( (6,0) \).  The figure and the
prior examples are illustrative, using up to \( 10 \) attacking armies
and \( 4 \) defending armies.  The R script below shows how to calculate
these quartiles for any number of attackers and defenders.

\begin{figure}
    \centering
    \includegraphics[scale=0.75]{risk_quartiles}
    \caption{Quartile statistics for the number of attacking armies
    remaining in various scenarios.}%
    \label{fig:riskgame:quartiles}
\end{figure}

The number of battles fought in a war is the number of dice rolls until
absorption into one of the states \( (a,0) \) or \( (0,d) \).  This is
the number of state transitions to absorption, also called the waiting
time until absorption.  The standard formula for calculating the
expected waiting time to absorption is \( (I-Q)^{-1} \mathbf{1} \).
Using this, the expected number of battles for a war with \( 10 \)
attacking armies against \( 4 \) defending armies is \( 3.994 \).  The R
script below shows how to calculate the expected waiting time to
absorption for any number of attackers and defenders.

\visual{Section Starter Question}{../../../../CommonInformation/Lessons/question_mark.png}
\section*{Section Ending Answer}

The order statistics are \( Y^{(3)} \), the largest value among \( Y_1,
Y_2, Y_3 \), \( Y^{(2)} \), the second largest value, including ties,
among \( Y_1, Y_2, Y_3 \), and \( Y^{(1)} \), the largest value,
including ties, among \( Y_1, Y_2, Y_3 \).

\subsection*{Sources} This section on \emph{Risk} is adapted from
\cite{osborne03},
\cite{pierce15} and
\cite{tan97}.

\hr

\visual{Algorithms, Scripts, Simulations}{../../../../CommonInformation/Lessons/computer.png}
\section*{Algorithms, Scripts, Simulations}

\subsection*{Algorithm}

\subsection*{Scripts}

%% \input{ _scripts}

\hr

\visual{Problems to Work}{../../../../CommonInformation/Lessons/solveproblems.png}
\section*{Problems to Work for Understanding}
\renewcommand{\theexerciseseries}{}
\renewcommand{\theexercise}{\arabic{exercise}}

\begin{exercise}
    Given the values \( Y_1 \), \( Y_2 \), \( Y_3 \) from a roll of \( 3
    \) dice, show that the probability for \( Y^{(1)} = y^{(1)} \), \( Y^
    {(2)} = y^{(2)} \) with \( y^{(1)} > y^{(2)} \) is \( \frac{6y^2 - 3}
    {216} \).
\end{exercise}
\begin{solution}
    For \( Y^{(1)} = y^{(1)} \), \( Y^{(2)} = y^{(2)} \) with \( y^{(1)}
    > y^{(2)} \), the dice rolls must be \( y^{(1)} \), \( y^{(2)} \)
    and \( y^{(3)} \le y^{(2)} \) \emph{in some order}.  There are \( 6 \)
    orders for each of the \( y^{(2)} \) possibilities for \( y^{(3)} \),
    each with probability \( \frac{1}{6^3} \).  However, if \( y^{(3)} =
    y^{(2)} \), then the \( 6 \) orders are counted twice, so half of
    them, specifically \( 3 \) must be subtracted.
\end{solution}

\begin{exercise}
    For the game of \emph{Risk}, show that the probability of the
    attacker rolling three dice, the defender rolling one die and the
    defender losing one army is \( 855/1296 \).  That is, calculate \(
    wxMaxim \rho_{310} \), the gray arrow in Figure~%
    \ref{fig:riskgame:risktrans} and Table~%
    \ref{tab:riskgame:risktransprobs}.
\end{exercise}
\begin{solution}
    Use the order statistic probability \( Y^{(1)} = (3y^2 - 3 y + 1)/216
    \).  The possibilities for the defender to lose one army are the
    following:
    \begin{enumerate}[label=(\alph*)]
    \item
        The largest die of the \( 3 \) rolled by the attacker is a \( 2 \),
        and the value rolled by the defender is \( 1 \); occurring with
        probability \( \frac{7}{216} \cdot \frac{1}{6} = \frac{7}{1296} \).
    \item
        The largest die of the \( 3 \) rolled by the attacker is a \( 3 \),
        and the value rolled by the defender is \( 1 \) or \( 2 \);
        occurring with probability \( \frac{19}{216} \cdot \frac{2}{6} =
        \frac{38}{1296} \).
    \item
        The largest die of the \( 3 \) rolled by the attacker is a \( 4 \),
        and the value rolled by the defender is \( 1 \) or \( 2 \) or \(
        3 \); occurring with probability \( \frac{37}{216} \cdot \frac{3}
        {6} = \frac{111}{1296} \).
    \item
        The largest die of the \( 3 \) rolled by the attacker is a \( 5 \),
        and the value rolled by the defender is \( 1 \), \( 2 \), \( 3 \),
        or \( 4 \); occurring with probability \( \frac{61}{216} \cdot
        \frac{4}{6} = \frac{244}{1296} \).
    \item
        The largest die of the \( 3 \) rolled by the attacker is a \( 6 \),
        and the value rolled by the defender is \( 1 \), \( 2 \), \( 3 \),
        \( 4 \), or \( 5 \); occurring with probability \( \frac{91}{216}
        \cdot \frac{5} {6} = \frac{455}{1296} \).
    \item
        The total probability is \( (7 + 38 + 111 + 244 + 455)/1296 =
        855/1296 \).
\end{enumerate}
\end{solution}

\begin{exercise}
    Find the expected number of attacking armies remaining when
    attacking an equal number of defending armies, for \( n = 20 \) to \(
    n = 1 \).
\end{exercise}
\begin{solution}

    \begin{tabular}{rrrrr}
        20     & 19       & 18          & 17           & 16          \\ 
        4.441  & 4.277    & 4.111       & 3.943        & 3.773       \\ 
        15     & 14       & 13          & 12           & 11          \\ 
        3.601  & 3.427    & 3.249       & 3.068        & 2.883       \\ 
        10     & 9        & 8           & 7            & 6           \\ 
        2.692  & 2.498    & 2.293       & 2.085        & 1.854       \\ 
        5      & 4        & 3           & 2            & 1           \\ 
        1.631  & 1.335    & 1.110       & 0.590        & 0.417
    \end{tabular}
\end{solution}

\hr

\visual{Books}{../../../../CommonInformation/Lessons/books.png}
\section*{Reading Suggestion:}

\bibliography{../../../../CommonInformation/bibliography}

%   \begin{enumerate}
%     \item
%     \item
%     \item
%   \end{enumerate}

\hr

\visual{Links}{../../../../CommonInformation/Lessons/chainlink.png}
\section*{Outside Readings and Links:}
\begin{enumerate}
    \item
    \item
    \item
    \item
\end{enumerate}

\section*{\solutionsname} \loadSolutions

\hr

\mydisclaim \myfooter

Last modified:  \flastmod

\end{document}

%%% Local Variables:
%%% TeX-master: t
%%% End:
